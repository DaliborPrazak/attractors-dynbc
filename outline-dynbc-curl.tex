\documentclass[12pt]{article}

\usepackage{amsmath} % 
\usepackage{amssymb} % 
\usepackage{amstext}
%\usepackage[notref,notcite]{showkeys}
\usepackage[utf8]{inputenc}
%\usepackage[czech]{babel}
\usepackage{verbatim}
\usepackage{xcolor}

\overfullrule 5pt

\parindent 0pt
\frenchspacing
\usepackage{a4wide}


%%%%%%%%%%%%%%%%%%%%%%%%%%%%%%%%%%%%%%%%%%%%%%%%%%%%%%%%%%%%
%% definice

\definecolor{alloy}{RGB}{196,98,16}
\definecolor{clue}{RGB}{59,157,239}

\newcommand{\BUNO}{\textcolor{blue}{WLOG}\ }

\newcommand{\uje}{u^{(1)}}
\newcommand{\ude}{u^{(2)}}

\newcommand{\vek}[1]{\pmb{#1}}
\newcommand{\parcd}[2]{\frac{\partial #1}{\partial #2}}
\newcommand{\ddy}{\frac{\partial^2}{\partial y^2}}
\newcommand{\sqlx}{\sqrt{\lambda+\xi^2}}
\newcommand{\sqls}{\sqrt{\lambda+s^2}}
\newcommand{\dy}{\parcd{}{y}}

\newcommand{\dert}{\partial_t}
\newcommand{\da}{\mathcal{D}(A)}
\newcommand{\diver}{\operatorname{div}}
\newcommand{\curl}{\operatorname{curl}}
\newcommand{\entau}[1]{[({#1})n]_{\tau}}
\newcommand{\rr}{\mathbb R}
\newcommand{\norm}[3]{\lVert #1 \rVert_{#2}^{#3}}
\newcommand{\ldvaom}{L^2(\Omega)}
\newcommand{\ldvagm}{L^2(\Gamma)}
\newcommand{\lpeom}{L^p(\Omega)}
\newcommand{\lpegm}{L^p(\Gamma)}
\newcommand{\intom}{\int\limits_\Omega}
\newcommand{\intgm}{\int\limits_\Gamma}
%\newcommand{\dentau}[1]{\left[(D{#1})n\right]_\tau}

%% H^{\infty} - calculus
\newcommand{\hhinf}{\mathcal{H}^{\infty}}
\newcommand{\cc}{\mathbb C}

%%% addenda
\newcommand{\attr}{\mathcal{A}}



\begin{document}
Let us consider the 2D NSE system with dynamic boundary condition
\begin{align}	\label{eq-1}
\dert u - \nu \Delta u + (u\cdot \nabla) u + \nabla \pi &= f
\qquad \diver{u} = 0 \qquad\textrm{in $\Omega$}
\\		\label{eq-2}
\beta \dert u + \alpha u + \nu \entau{Du} &= \beta h
\qquad u\cdot n = 0 \qquad\textrm{on $\Gamma = \partial \Omega$}
\end{align}
where $\Omega \subset \rr^2$ is a suitable planar domain.
It was shown in \cite{PZ24}, given that $\Omega$ is smooth
and {\it bounded}, the fractal dimension of the global attractor can
be estimated as
\begin{equation}	\label{es-1}
\dim_{H}(\attr) \le c_0 \frac{M_\beta}{m^{3/2}_\alpha}
\cdot \frac{\ell^2\norm{F}{H}{}}{\nu^2}
\end{equation}
where $F=(f,h)$, $H=L^2(\Omega \times \Gamma)$, $
\ell \sim \operatorname{diam}\Omega$ is the characteristic
length  and
\begin{equation}	\label{defMm}
m_\alpha = \min\{1 , \alpha \ell / \nu \}
\qquad
M_\beta = \max\{ 1 , \beta / \ell \}
\end{equation}
One observes that the last term in \eqref{es-1} corresponds
to the so-called Grashof number $G = \norm{F}{H}{} / \lambda_1 \nu^2$.
For $\alpha \to \infty$, $\beta \to \infty$, 
boundary condition \eqref{eq-2} reduces to homogeneous Dirichlet $u=0$,
while \eqref{es-1} reduces to $\dim(\attr) \le G$, which is consistent
with the best available estimate in this setting, see \cite{Te97}.

\par\medskip
{\bf Question.} What about the limit $\alpha$, $\beta \to 0$?
This would reduce \eqref{eq-2} to $\entau{Du} = 0$, where one
has an improved estimate
\begin{equation}	\label{es-2}
\dim(\attr) \le c_1 G^{2/3}
\end{equation}
(up to some logarithmic term); see \cite{Il94}, \cite{Zi98}. 
Note that \eqref{es-2} also holds in case of periodic boundary
conditions, see \cite{Te97}. Unfortunately, our estimate \eqref{es-1} 
blows up for for $\alpha$ small, so a different approach is needed.

%%%%%%%%%%%%%%%%%%%%%%%%%%%%%%%%%%%%%%%%%%%%%%%%%%%%%%%%%%%%
\par\medskip
{\bf Some ideas.} Let us work with $\omega = \curl u$, which is
motivated by two facts. Firstly, $\entau{Du} = \omega$ on $\Gamma$,
at least for the flat boundary. Secondly, the estimate \eqref{es-2}
is proved while working with $u\in W^{1,2}(\Omega)$, or testing the equation
with $Au = -\Delta u$. This is equivalent to working with
$\omega \in L^2$, or testing the equation for $\omega$ by $\omega$. 
\par
So, let us apply $\curl$ to \eqref{eq-1}, to get
\begin{align}	\label{cu-1}
\dert \omega - \nu \Delta \omega + (u\cdot \nabla)\omega &= \curl f
\qquad\textrm{in $\Omega$}
\\      \label{cu-2}
\beta \dert u + \alpha u + \nu \omega &= \beta h
\qquad\textrm{on $\Gamma = \partial \Omega$}
\end{align}
Now we have several tasks:
\begin{enumerate}
\item Prove the estimate \eqref{es-2} for this system if
$\alpha$, $\beta=0$. This seems to work, see a sketch
(TODO) below.

\item Show that somehow the estimate is robust 
also for $\alpha$, $\beta$ very small, maybe via some
asymptotic similarity of both systems?

\item As previous taks is presumably hard, we could
also look at some further simplification. For example,
we ignore the interior evolution completely, that is to
say, to replace \eqref{cu-1} by
\begin{equation}	\label{eq-1st}
- \nu \Delta u + (u\cdot \nabla) u + \nabla \pi = f
\end{equation}
or even take $f=0$. This makes sense since for $\beta$ small,
the boundary evolution is much faster than the interior one.

\item Last, but not least: prove just the existence of global
attractor for (\ref{eq-1}--\ref{eq-2}) in case of $\Omega$
{\it unbounded}. This is not trivial because of problems 
with both dissipativity (no Poincaré inequality)
and asymptotic compactness (unbouded domain).

\end{enumerate}




\begin{thebibliography}{10}

%\bibitem{AACG18} Acevedo, Amrouche, Conca, Ghos. 
%C. R. Math. Acad. Sci. Paris, 2019. DOI 10.1016/j.crma.2018.12.002 
%\bibitem{AG18} Amrita Ghosh, Ph.D. thesis, 2018.
%\bibitem{EM19} Erika Maringová, Ph.D. thesis, 2019.
\bibitem{Il94} Ilyin, A.A.: Partly dissipative semigroups generated by the Navier-Stokes system on two-dimensional manifolds, and their attractors. Russ. Acad. Sci., Sb., Math. 78(1), 47–76 (1994).
\bibitem{PZ24} D. Pražák, M. Zelina: Strong solutions and attractor dimension for 2D NSE with dynamic boundary conditions. J. Evol. Equ. 24, 20 (2024). 
\bibitem{Te97} Temam, R.: Infinite-dimensional dynamical systems in mechanics and physics, Applied Mathemat- ical Sciences, vol. 68, second edn. Springer-Verlag, New York (1997).
\bibitem{Zi98} Ziane, M.: On the two-dimensional Navier-Stokes equations with the free boundary condition. Appl. Math. Optim. 38(1), 1–19 (1998).


\end{thebibliography}



\end{document}
